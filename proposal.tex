\documentclass[a4paper, 11pt]{article}

\usepackage{kotex} % Comment this out if you are not using Hangul
\usepackage{fullpage}
\usepackage{hyperref}
\usepackage{amsthm}
\usepackage[numbers,sort&compress]{natbib}

\theoremstyle{definition}
\newtheorem{exercise}{Exercise}

\begin{document}
%%% Header starts
\noindent{\large\textbf{IS-521 Activity Proposal}\hfill
                \textbf{한영광}} \\
         {\phantom{} \hfill \textbf{juanaevv}} \\
         {\phantom{} \hfill Due Date: April 15, 2017} \\
%%% Header ends

\section{Activity Overview}

본 activity는 packet 변조 공격의 잠재적 위험성과 여러 활용법에 대해 알아보고 Intrusion Detection System의 대략적인 개념을 code에 적용하여 특정한 패킷의 경우 connection을 받지 않는 IDS가 적용된 server 제작을 통해 Network security의 기초적인 공방을 체험해 본다.
\emph{why}
최근, Iot 및 오토카 등이 인간 생활과 밀접한 연관이 생길 가능성이 증가 되고 있는 상황에서 네트워크 통신에서 Dos attack에 사용될 수 있는 bot의 후보 역시 늘어나고 있는 추세이다. 이러한 상황에서 Iot 기술이 적용된 사물에 Dos 공격 역시 쉬워질 것이고 이는 인간의 생활에도 큰 영향을 미칠 가능성이 늘어난다. 이러한 환경에서 변조된 network에 대한 공격 역시 유행이 될 것이기에 본 activity를 제안합니다.

\section{Exercises}

Describe a series of exercises that students will carry out. (학생들이 하게
될 연습문제를 순차적으로 서술.)

\begin{exercise}

  Raw socket coding을 통하여 간단한 특정환경이 설정된 IDS가 코딩된 서버 프로그램에 대한 연결을 성공하는 것을 목적으로 attack client 코딩을 요구한다. 여기서 server 프로그램에 대한 소스코드는 제공한다.

\end{exercise}

\begin{exercise}

  본인이 설계한 attack client에 대한 침입을 감지하여 connection 허용을 하지 않는 server를 기본 제공된 server 소스코드에 추가적인 코딩을 통해 제작하도록 한다.

\end{exercise}

\begin{exercise}

  TCP SYN flooding을 발생하는 client program과 이를 방어하도록 하는 server 프로그램을 앞에서 제공된 server 소스코드에 추가적으로 코딩 하여 제작한다. 실험환경은 localhost로 하도록 한다.

\end{exercise}

\section{Expected Solutions}

\begin{solution}
적용된 IDS의 종류 (IP based, Protocol based)에 따라 학생들은 다양한 raw socket coding program을 구현할 수 있다. 학생들은 본인의 공격에 맞는 대응법이 적용된 server를 제작할 수 있다. SYN flooding의 원리를 이해할 수 있고 server 측에선 connect 맺을때 많은 시간당 request를 보내는 ip에 대해 거부하는 프로그램을 제작할 수 있다.
\end{solution}

\bibliography{references}
\bibliographystyle{plainnat}

\end{document}
